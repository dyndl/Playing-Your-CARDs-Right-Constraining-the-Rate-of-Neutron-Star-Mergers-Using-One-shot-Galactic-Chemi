\begin{abstract}
The recent gravitational wave detection of a neutron star merger using the Laser Interferometer Gravitational Wave Observatory and virgo?? (VIRGO) has been used to help constrain the neutron star merger rate in the local universe. However, another way to constrain the rates is to assess the $\alpha$ and $r$-process elemental abundance budgets granted by the very brief star formation histories and the subsequent core-collapse supernova (SN) and neutron star merger (NSM) events, found in UFDs. Here we utilize an ensemble of chemical abundance ratio distribution (CARD) models of ultra-faint dwarf (UFD) galaxies to constrain the probable neutron star merger and supernova rates in UFDs. We find that the CARDs of different elements like, e.g., Barium and Magnesium, can be well approximated by assuming that the star formation histories in UFDs and progenitors of the very metal-poor (VMP) Milky Way stellar halo (MWH) are comprised of brief, `{\it one-shot}' events. In our analysis, we find that the NSM-to-SN rate, R$_{NSM/SN}$, most consistent with current CARD data in UFDs and the VMP MWH is ??0.002$^{\pm0.0003}$??. This rate corresponds to an NSM rate of ???? assuming that the average SN rate in the early universe is ????. From our analysis, we also get estimates for the star formation efficiency, blowout fraction of metals and gas, and level of metal mixing in UFDs and find that these values are consistent with those found in the literature by other means.
}