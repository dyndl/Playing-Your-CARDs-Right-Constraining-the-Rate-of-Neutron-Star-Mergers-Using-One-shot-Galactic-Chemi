\section{Introduction}
\label{intro} 
The recent detection and confirmation of a neutron star merger (NSM) using the Laser Interferometer Gravitational Wave Observatory (LIGO) and virgo?? (VIRGO) along with a host of other great ground and space-based observatories across the electromagnetic spectrum has led to a variety of new constraints on phenomena ranging from neutron star equations-of-state to the nature of the fabric of spacetime itself(?). In particular, the detection places new constraints on the NSM rates (R$_{NSM}$) found in the local universe. 

Constraining R$_{NSM}$ is important because of the role NSMs have along with core-collapse supernovae (SN) in galactic chemical evolution and in the precise 'odd-even' elemental abundance patterns found in the Sun and in stars that are much more metal poor. These patterns originate from the elemental yields produced from the stellar lifecycle during stellar nucleosynthesis (B2FH?), stellar evolution (especially in the asymtotic giant branch (AGB) phase)(FHerwig), explosive stellar nucleosynthesis (i.e., CCSN), and $r$-process-rich nucleosynthetic events from stellar remnants (i.e., NSM and rare-CCSN). 

%\begin{equation}
%\label{eqn:drag}
%	\int_a^bu\frac{d^2v}{dx^2}\,dx
%	=\left.u\frac{dv}{dx}\right|_a^b
%	-\int_a^b\frac{du}{dx}\frac{dv}{dx}\,dx.
%\end{equation}
