\section{Introduction}
\label{intro} 
The recent detection and confirmation of a neutron star merger (NSM) using the Laser Interferometer Gravitational Wave Observatory (LIGO) and virgo?? (VIRGO) along with a host of other great ground and space-based observatories across the electromagnetic spectrum has led to a variety of new constraints on phenomena ranging from neutron star equations-of-state to the nature of the fabric of spacetime itself(?). In particular to our study, the detection places new constraints on the NSM rates (R$_{NSM}$) found in the local universe. 

Constraining R$_{NSM}$ is important because of the role NSMs have along with core-collapse supernovae (CCSN) in galactic chemical evolution (GCE) and

in the precise ``odd-even" elemental abundance patterns found in the Sun and in stars that are relatively metal deficient. These patterns originate from the elemental yields produced from the stellar lifecycle during stellar nucleosynthesis (e.g., main-sequence core burning; $\alpha$-process) (B2FH?), post main-sequence stellar evolution (e.g., in the asymtotic giant branch (AGB) phase; $s$-process) (F.Herwig, ??), explosive stellar nucleosynthesis (e.g., CCSN; $i$-process, $weak$ $r$-process...) (???), and $r$-process-rich nucleosynthetic events from stellar remnants (i.e., NSM and $rare$-CCSN). In regards to GCE, improved accuracy on both the rate and yields of NSMs and CCSNs improve the ability of modelers and simulators of galaxy evolution to accurately recreate the effects that dynamical and nucleosynthetic processes have on the distribution and processing of gas, metals, stars, and dark matter in various galactic systems.

****
Add Josh, Evan's, Anna's and Alex's work on growing the sample, asking questions about the meaning of the distributions and demonstrating the distributions... Figure of compliation of data [for Anna!]
****

intro UFD
which have total masses $\lesssim10^7 M_\odot$
leading to short-lived, ``{\it one-shot}" star formation histories (SFHs) via {\it in situ} stellar feedback and other galactic environmental factors (e.g., ram-pressure stripping, tidal stripping, galactic harassment, reionization; refs???). 


In this work, we aim to use the unique information provided by [XX] metal-poor stars in ultra-faint dwarfs (UFDs) and in the Milky Way halo.

(which likely originate from early-accreted UFD-like analogs). 

most(?) of the stars with [Fe/H] $\lesssim$ 2.5 dex originate from the early galactic systems that themselves formed in the early Universe and that are believed to have been the building blocks of the MW.



As {\it one-shot} galaxies, mostxx UFDs are expected to have only few episodes of SF, 
that lead to a sligthly metal-enriched interstellar medium (ISM) enrichment

by which generations of stars are enriched and then passively evolved to the present (Brown, Tumlinson...). 

To model these systems, we extend the CARD modeling technique established in \citet{Lee_2013} to include NSM enrichment. These new {\it one-shot} models can be used to compare to the data as representative of the {\it parent} distributions from which CARD sample are drawn. We also extend the \citet{Lee_2013} analysis to explicitly derive certain galactic stellar activity and {\it in situ} environmental parameters, e.g., average star formation efficiency ($\left<SF_{eff}\right>$) and the average fraction of CCSN and NSM metals ejected from UFDs ($\left<f_{ej,CCSN}\right>$, $\left<f_{ej,NSM}\right>$) for each CARD model. In \S\ref{methods}, we the discuss how we generate our models, how we select samples of data for comparison to the models, and how we determine a goodness-of-fit for each model to the data. In \S\ref{results}, we present the results of our analysis and compare our findings to those in the literature. In \S\ref{discuss}, we place our results the context of current and upcoming surveys. In \S\ref{conc}, we summarize our analysis and results and make predictions for testing the robustness of our results.