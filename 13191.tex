\section{Introduction}
\label{intro} 
The recent detection and confirmation of a neutron star merger (NSM) using the Laser Interferometer Gravitational Wave Observatory (LIGO) and virgo?? (VIRGO) along with a host of other great ground and space-based observatories across the electromagnetic spectrum has led to a variety of new constraints on phenomena ranging from neutron star equations-of-state to the nature of the fabric of spacetime itself(?). In particular, the detection places new constraints on the NSM rates (R$_{NSM}$) found in the local universe. 

Constraining R$_{NSM}$ is important because of the role NSMs have along with core-collapse supernovae (CCSN) in galactic chemical evolution (GCE) and in the precise ``odd-even" elemental abundance patterns found in the Sun and in stars that are relatively metal deficient. These patterns originate from the elemental yields produced from the stellar lifecycle during stellar nucleosynthesis (e.g., main-sequence core burning; $\alpha$-process) (B2FH?), post main-sequence stellar evolution (e.g., in the asymtotic giant branch (AGB) phase; $s$-process) (F.Herwig, ??), explosive stellar nucleosynthesis (e.g., CCSN; $i$-process, $weak$ $r$-process...) (???), and $r$-process-rich nucleosynthetic events from stellar remnants (i.e., NSM and $rare$-CCSN). In regards to GCE, improved accuracy on both the rate and yields of NSMs and CCSNs improve the ability of modelers and simulators of galaxy evolution to accurately recreate the effects that dynamical and nucleosynthetic processes have on the distribution and processing of gas, metals, stars, and dark matter in various galactic systems.

In this work, we aim to use the unique information made available to us in very metal-poor (VMP) galactic systems; namely, ultra-faint dwarfs (UFDs) and the VMP Milky Way halo (MWH) stars (which likely originate from early-accreted UFD-like analogs). In these systems, most(?) of the stars that we see below [Fe/H] $\sim$ 2.5 dex originate from the first galactic systems in the Universe which have total masses $\lesssim10^7 M_\odot$ leading to short-lived, ``{\it one-shot}" star formation histories (SFHs) via {\it in situ} stellar feedback and other galactic environmental factors (e.g., ram-pressure stripping, tidal stripping, galactic harassment, reionization; refs???). As {\it one-shot} galaxies, most UFDs are expected to have $\sim$2--3 episodes of interstellar medium (ISM) enrichment by which generations of stars are enriched and then passively evolved to the present (Brown, Tumlinson...). To model these systems, we extend the CARD modeling technique established in \cite{Lee_2013} to treat the observed stars as a {\it fair} sample of stellar chemical abundances drawn from {\it parent} enriching stellar generations (ESGs). In \cite{Lee_2013} an emsemble of $n$ ESGs of a given mass, M$_{ESG}$
\footnote{Since the models do not invoke a truncated IMF, M$_{ESG}^0$ $\simeq$ M$_{ESG}^{draw}$ with $\sigma_{ESG}$ $\sim$ ???}, 
are the approximate total mass of ESG  stochastically sampled from a Salpeter IMF \citep{Salpeter_1955} in each ESG represents the mass $n$


In \cite{Lee_2013} these ESGs are defined by the CARDs by stochastically sampling stars from a Salpeter IMF \citep{Salpeter_1955} to    

Both sample of stars can 