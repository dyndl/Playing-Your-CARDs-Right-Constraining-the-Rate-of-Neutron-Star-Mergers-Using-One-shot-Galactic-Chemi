\section{The recent gravitational wave detection of a neutron star merger using the Laser Interferometer Gravitational Wave Observatory and virgo?? (VIRGO) has been used to help constrain the neutron star merger rate in the local universe. However, another way to constrain the rates is to assess the $\alpha$ and $r$-process elemental abundance budgets granted by the very brief star formation histories and the subsequent core-collapse supernova (SN) and neutron star merger (NSM) events, found in UFDs. Here we utilize an ensemble of chemical abundance ratio distribution (CARD) models of ultra-faint dwarf (UFD) galaxies to constrain the probable neutron star merger and supernova rates in UFDs. We find that the CARDs of different elements like, e.g., Barium and Magnesium, can be well approximated by assuming that the star formation histories in UFDs and progenitors of the very metal-poor (VMP) Milky Way stellar halo (MWH) are comprised of brief  ('$one-shot$') events. In our analysis, we find that the NSM-to-SN rate, R$_{NSM/SN}, most consistent with current CARD data in UFDs and the VMP MWH is ??0.002$^{\pm0.0003}??. This rate corresponds to an NSM rate of ???? assuming that the average SN rate in the early universe is ????. From our analysis, we also get estimates for the star formation efficiency, blowout fraction of metals and gas, and level of metal mixing in UFDs and find that these values are consistent with those found in the literature by other means.Method}
\label{methods}





The analysis and methods used in this paper follow and extend the methods used in \citet{Lee_2013}. In that paper, \citet{Lee_2013} proposed a novel approach to modeling the CARDs observed in both the MW halo and in UFD stars. Instead of chemically tracing the time evolution of the average chemical abundance ratio in their UFDs (see, e.g., \citet{Cohen_2010}, \citet{?}, and \citet{?}; i.e., the standard GCE modeling method), \citet{Lee_2013} choose to eliminate the explicit tracking of time in their GCE modeling to create models capable of constraining both the GCE in UFDs {\it and} the underlying nucleosynthetic yields of CCSN in such systems---a method similar to the one first proposed in \citet{Karlsson2005a} and demonstrated in \citet{Karlsson2005b}. In particular, the method generates UFD models that represent the {\it parent} probability density functions of CARDs found in those galaxies (and coincidently in the VMP MW halo). To do this, 


As state in \S\ref{intro}, UFDs experience only a few chemically-distinct enrichment episodes due to their SFHs. In doing so, UFDs provide the simplest galactic environments for understanding GCE, nucleosysthesis in all forms, and the precise origin of the elements. 


Instead of focusing on tracing the time evolution in the short epoch of star formation in UFDs, they decided to 



The analysis treats the observed stars as a {\it fair} sample of stellar chemical abundances drawn from {\it parent} enriching stellar generations (ESGs). In \citet{Lee_2013} an ensemble of ESGs, $n_{ESG}$, of a given stellar mass, M$_{ESG}$
\footnote{Since the models do not invoke a truncated IMF, M$_{ESG}^0$ $\simeq$ M$_{ESG}^{draw}$ with $\sigma_{ESG}$ $\sim$ ???}, 
are stochastically sampled from a Salpeter IMF \citep{Salpeter_1955}. The total elemental mass yields of various elements tracked in the models are used to derive CARDs by converting the mass yields of each ESG realization in an ensemble of $n_{ESG} = 10^4$ to chemical abundances measurements. The distribution of chemical abundances in the ensemble constitute a {\it closed-box} statistical model representation of chemical enrichment in {\it one-shot} galaxies. 


In \citet{Lee_2013} these ESGs are defined by the CARDs by stochastically sampling stars from a Salpeter IMF \citep{Salpeter_1955} to    

Both sample of stars can 