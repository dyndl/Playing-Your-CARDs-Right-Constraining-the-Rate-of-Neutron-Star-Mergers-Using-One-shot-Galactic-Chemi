\section{Method and Analysis}
\label{methods}
The analysis and methods used in this paper follow and extend the methods used in \citet{Lee_2013}. In that paper, \citet{Lee_2013} proposed a novel approach to modeling the CARDs observed in both the MW halo and in UFD stars. Instead of chemically tracing the time evolution of the average chemical abundance ratio in their UFDs (see, e.g., \citet{Cohen_2010}, \citet{?}, and \citet{?}; i.e., the standard GCE modeling method), \citet{Lee_2013} choose to eliminate the explicit tracking of time in their GCE modeling to create models capable of constraining both the GCE in UFDs {\it and} the underlying nucleosynthetic yields of CCSN in such systems---a method similar to the one first proposed in \citet{Karlsson2005a} and demonstrated in \citet{Karlsson2005b}. In particular, their method generates UFD models that represent the {\it parent} probability density functions of CARDs (generated by CCSN) found in those galaxies (and in the VMP MW halo by extension). Our work here creates increases the  statistical robust of the models by increasing the number of ESG realizations generated for each model (see \citet{Lee_2013} for details) and extends the previous models to {\it explicitly} include NSM enrichment.

\subsection(CARDs in UFDs)
\label{ufdcards}
As state in \S\ref{intro}, UFDs experience only a few chemically-distinct enrichment episodes due to their SFHs. In doing so, UFDs provide the simplest galactic environments for understanding GCE, nucleosysthesis in all forms, and the precise origin of the elements. 




\subsection(The models)
\label{models}
The analysis treats the observed stars as a {\it fair} sample of stellar chemical abundances drawn from {\it parent} enriching stellar generations (ESGs). In \citet{Lee_2013} an ensemble of ESGs, $n_{ESG}$, of a given stellar mass, M$_{ESG}$
\footnote{Since the models do not invoke a truncated IMF, M$_{ESG}^0$ $\simeq$ M$_{ESG}^{draw}$ with $\sigma_{ESG}$ $\sim$ ???}, 
are stochastically sampled from a Salpeter IMF \citep{Salpeter_1955}. The total elemental mass yields of various elements tracked in the models are used to derive CARDs by converting the mass yields of each ESG realization in an ensemble of $n_{ESG} = 10^4$ to chemical abundances measurements. The distribution of chemical abundances in the ensemble constitute a {\it closed-box} statistical model representation of chemical enrichment in {\it one-shot} galaxies. 


In \citet{Lee_2013} these ESGs are defined by the CARDs by stochastically sampling stars from a Salpeter IMF \citep{Salpeter_1955} to    

\subsection(The Data)
\label{data}
Both sample of stars can 