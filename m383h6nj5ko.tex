\section{Methods}
\label{methods}

The analysis and methods used in this paper follow and extend the methods used in \citet{Lee_2013}. In that paper, \citet{Lee_2013} proposed a novel approach to modeling the CARDs observed in both MW halo and UFD stars. Instead of chemically tracing the time evolution of the average chemical abundance ratio in their UFDs \citep[see, e.g.,][i.e., a standard GCE modeling method]{Cohen_2010,???,???reviews}, \citet{Lee_2013} choose to eliminate the explicit tracking of time evolution in GCE modeling to create models capable of constraining both the "broa" GCE the underlying nucleo  


Instead of focusing on tracing the time evolution in the short epoch of star formation in UFDs, they decided to 



The analysis treats the observed stars as a {\it fair} sample of stellar chemical abundances drawn from {\it parent} enriching stellar generations (ESGs). In \citet{Lee_2013} an ensemble of ESGs, $n_{ESG}$, of a given stellar mass, M$_{ESG}$
\footnote{Since the models do not invoke a truncated IMF, M$_{ESG}^0$ $\simeq$ M$_{ESG}^{draw}$ with $\sigma_{ESG}$ $\sim$ ???}, 
are stochastically sampled from a Salpeter IMF \citep{Salpeter_1955}. The total elemental mass yields of various elements tracked in the models are used to derive CARDs by converting the mass yields of each ESG realization in an ensemble of $n_{ESG} = 10^4$ to chemical abundances measurements. The distribution of chemical abundances in the ensemble constitute a {\it closed-box} statistical model representation of chemical enrichment in {\it one-shot} galaxies. 


In \citet{Lee_2013} these ESGs are defined by the CARDs by stochastically sampling stars from a Salpeter IMF \citep{Salpeter_1955} to    

Both sample of stars can 